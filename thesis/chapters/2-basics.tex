\section{Basics}
\label{sec:basics}

There are certain terms and concepts the reader should be familiar with in order to understand the aim and purpose of this thesis. 
We forego a formal mathematical introduction to relational algebra, although a familiarity of its basic concepts is highly recommended. 
Moreover, we assume that the Landau notation is known.

\subsection{Relations}
Relations are tables in a database. 
Its attributes are described by a database schema, which specifies the possible attribute types.
An entry in the table is described as a \textit{tuple}. Furthermore, we call the number of tuples in a relation the \textit{cardinality} of the relation.

\subsection{Join}
A join links two or more relations by building their cross product and filtering it with the help of a \textit{join predicate}. 
The result is a new relation which contains the joined tuples, for which the predicate is true. We denote the join operator using the symbol "$\Join$". 
As an example, we denote the corresponding join for the two relations $R$ and $S$ with ``''$R \Join S$''.

\subsection{Selectivity}
The selectivity of two relations is the ratio between the cardinality of their cross product and the number of elements in the join's resulting relation. 
This way we determine the number of entries that is kept after the join.

For two relations $R_i$ and $R_j$ regarding a predicate $p_{i,j}$ the selectivity is thus given by 
\begin{equation}
f_{i,j} = \frac{|R_i\Join_{p_{i,j}}R_j|}{|R_i \times R_j|}	
\end{equation}

Since we can assume that $|R_i\Join_{p_{i,j}}R_j| \geq 0$, $|R_i \times R_j| > 0$ and $|R_i\Join_{p_{i,j}}R_j| \leq |R_i \times R_j|$ we know that $f_{i,j} \in [0,1]$.

For simplification purposes we only inspect joins with at most one join predicate. Hence, there is just one selectivity between two relations given. 
\newpage

\subsection{Query Types}
The relations can be connected in many different ways.
We visualize this connection in a graph.
There is a number of different query types. 
Most importantly we want to have a look at the following four query types, because they represent distinct problems:

\begin{figure}[htp]
\begin{center}
\begin{subfigure}[c]{.3\textwidth}
\vspace{0pt}
\centering
\psmatrix[colsep=0.5cm,rowsep=0.5cm,mnode=circle]
$R_1$\\
& $R_2$\\
& & $R_3$
\ncline{-}{1,1}{2,2}
\ncline{-}{2,2}{3,3}
\endpsmatrix
\subcaption{Chain}
\end{subfigure}
\hspace{2cm}


\begin{subfigure}[c]{.3\textwidth}
    \psmatrix[colsep=2cm,rowsep=2cm,mnode=circle]
    $R_1$ & $R_2$\\
    $R_3$ & $R_4$
    \ncline{-}{1,1}{1,2}
    \ncline{-}{2,1}{2,2}
    \ncline{-}{1,1}{2,1}
    \ncline{-}{1,2}{2,2}
    \endpsmatrix
    \subcaption{Circle}
    \end{subfigure}
    \begin{subfigure}[c]{0.3\textwidth}
    \vspace{0.8cm}
    \psmatrix[colsep=0.5cm,rowsep=0.5cm,mnode=circle]
    $R_1$ & & $R_3$\\
    & $R_2$\\
    $R_4$ & & $R_5$
    \ncline{-}{1,1}{2,2}
    \ncline{-}{1,3}{2,2}
    \ncline{-}{3,1}{2,2}
    \ncline{-}{3,3}{2,2}
    \endpsmatrix 
    \subcaption{Star}
    \end{subfigure}
    \hspace{2cm}
    \begin{subfigure}[c]{0.3\textwidth}
    \vspace{0.5cm}
    \psmatrix[colsep=0.5cm,rowsep=0.5cm,mnode=circle]
    & $R_1$\\
    $R_2$ & & $R_3$\\
    $R_4$ & & $R_5$
    \ncline{-}{2,1}{1,2}
    \ncline{-}{2,3}{1,2}
    \ncline{-}{3,1}{1,2}
    \ncline{-}{3,3}{1,2}
    \ncline{-}{2,1}{2,3}
    \ncline{-}{2,1}{3,3}
    \ncline{-}{2,1}{3,1}
    \ncline{-}{3,1}{3,3}
    \ncline{-}{2,3}{3,3}
    \ncline{-}{3,1}{2,3}
    \endpsmatrix
    \subcaption{Clique}
    \end{subfigure}
    \end{center}
    \caption{Query Graphs}
    \end{figure}

\subsection{Dynamic Programming}
    
