\section{Concepts}

\subsection{Domain-Specific Algorithms}

\subsubsection{DPCPP}

\subsubsection{Adaptive Radix Tree}

\subsection{Server}

In order for the server not having to memorize client or session information, we want to establish a stateless communications protocol.
Thus, we're conforming to the Representational state transfer (REST) architecture as laid out in \cite{fielding2000architectural}.
The exact API entry points are specified in \ref{sub:api-entry-points}.

\subsubsection{API Entry Points}
\label{sub:api-entry-points}

\subsection{Client}

\subsubsection{Query Graphs}

Visual representations of query graphs are drawn on a canvas with a top-left origin.
With $n$ we define the number of nodes, $h$ is the height of the canvas, and $w$ its width. 
As restricted by the user interface, we also set the condition that $n \geq 3$.

Additionally, we define $m$ as the margin to the canvas bounds, and $r$ as the radius of a node.
Taking $m$ and $r$ into consideration, the drawable width of the canvas, i.e. a centered layout box inside the canvas with margin $m$, is given by 

\begin{equation}\label{eqn:painting-w_hat}
    \hat{w} = w - 2(m - r)
\end{equation}

and analogously, the drawable height is given by

\begin{equation}\label{eqn:painting-h_hat}
    \hat{h} = h - 2(m - r)    
\end{equation}

\paragraph{Chain} For chain query graphs we draw nodes at the vertical center of the canvas and set an equal spacing in between. Thus, the canvas point ($x_i$, $y_i$) for the node with index $i$ is given by


\begin{equation}
    (x_i, y_i) = (m + r + \frac{i\hat{w}}{n - 1}, \frac{h}{2})
\end{equation}

\paragraph{Star} 
Star query graphs are drawn by placing the node with index $i = 0$ at the canvas' center, and all following nodes in a circle around it, starting at the rightmost vertically centered point ($\theta_0 = 0^{\circ}$).

The radius of this circle is equivalent to %Cycle graph%
and given by 

\begin{equation}\label{eqn:painting-r_star}
    r_{star} = \frac{w}{2} - r - m
\end{equation}

For the current node we define the angle by
\begin{equation}\label{eqn:painting-theta}
    \theta_i = \frac{2i\pi}{n - 1}
\end{equation}

Thus, the point for the node at index $i$ is given by

\begin{equation}
    (x_i, y_i) = 
    \begin{cases}
        (\frac{w}{2}, \frac{h}{2}),& \text{if } i = 0\\
        (r_{star}\cos{\theta_i} + r_{star} + r + m, r_{star}\sin{\theta_i} + r_{star} + r + m), & \text{otherwise}
    \end{cases}
\end{equation}

\paragraph{Tree}

The node position calculation for tree query graphs requires an additional parameter. 
By $k$ we denote the degree of the tree. So far only complete $k$-ary query graphs be drawn.
First, we calculate the vertical position for each given node $i$.

\begin{equation}
    (x_i, y_i) = (i - 2^{\lfloor \log_k(i+1) \rfloor}, r + m + \frac{\hat{w}}{\lfloor \log_kn \rfloor} \lfloor \log_k(i + 1) \rfloor)
\end{equation}

\paragraph{Cycle} 

We define $r_{star}$ the same way as we did in \eqref{eqn:painting-r_star}.
Similarly to \eqref{eqn:painting-theta}, $\theta_i$ is given by

\begin{equation}\label{eqn:painting-theta_cycle}
    \theta_i = \frac{2i\pi}{n}
\end{equation}

Thus, the point for a respective node is given by

\begin{equation}
    (x_i, y_i) = (r_{star}\cos{\theta_i} + r_{star} + r + m, r_{star}\sin{\theta_i} + r_{star} + r + m)
\end{equation}