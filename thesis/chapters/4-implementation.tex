\section{Implementation}

\subsection{API Requirements and Design Goals}
One of the main goals of this framework is to minimize the amount of code or configuration that has to be added
to exisiting algorithms' implementations.

More formally, this means that the algorithm has to conform to the \texttt{Visualizable} type we define as follows:

\begin{minted}{go}
    type Visualizable func(QG QueryGraph)
\end{minted}


\subsection{Choice of Technology}

For the implementation of both client and server code we want to maximize

\begin{enumerate}[label={(\roman*)}]
    \item \label{itm:accessibility} \textbf{Accessibility}, a minimal amount of effort required to access the application
    \item \label{itm:maintainability} \textbf{Maintainability}, a high degree of technological similarities with related projects 
    \item \label{itm:testedness} \textbf{Tried-and-testedness}, a large user base and well-received reference applications
\end{enumerate}

In order to maximize \ref{itm:accessibility} we create an application that runs in a web browser and doesn't need any user-initiated installation or update procedures.
With the 2017 released WebAssembly (WASM)\footnote{\url{https://webassembly.org}} portable byte-code standard still being in its infancy at around 90\% browser support\footnote{\url{https://caniuse.com/wasm}} and being recommended by the World Wide Web Consortium not until December 5th 2019\footnote{\url{https://www.w3.org/TR/wasm-core-1/}}, this leaves us with JavaScript as the only viable language for the client-side code. 

Additionally, in order to modularize the user interface and adding the ability to import existing user interface components, we use \texttt{React.js}\footnote{\url{https://reactjs.org}}, as of writing this thesis the most popular JavaScript library for building user interfaces.
%\footnote{Ranked by number of stars on \url{https://github.com/search?l=JavaScript&o=desc&q=language%3AJavaScript&s=stars&type=Repositories}}.
% search?l=Go&o=desc&q=http&s=stars&type=Repositories

For the server code and algorithm implementations we use the Go programming language and \texttt{gin-gonic}\footnote{\url{https://github.com/gin-gonic/gin}}, which at the time of writing this thesis is the most-starred HTTP framework on GitHub \footnote{\url{https://github.com/}}.

%Thus, we resort to modern, but well-adopted languages and libraries.
%The software should be built with technologies that prepare well for continued development and maintenance.

\subsection{Utility Functions}
In order to provide a starting point for the algorithm implementation, we use a modified and extended version of the \url{infra.go} and \url{joinproblemparser.go} file provided for the exercises in the Query Optimization (CS50x) course \texttt{} at the Chair of Practical Computer Science III at the University of Mannheim.

We add some additional helper functions:
\begin{minted}{go}   
    func contains(s []uint, e uint) bool {
        for _, a := range s {
            if a == e {
                return true
            }
        }
        return false
    }    
\end{minted}

\subsection{Linting}
The project uses the Go language linter\footnote{\url{https://github.com/golang/lint}} to ensure a common coding and documentation style with what is used at Google, the creators of Go.

For the client-side code we use ESLint\footnote{\url{https://eslint.org}} configured with the \texttt{"react-app"} template\footnote{\url{https://github.com/facebook/create-react-app/blob/master/.eslintrc.json}}, which itself is an extension of the \texttt{"eslint:recommended"} template as specified on \url{https://eslint.org/docs/rules/}.

\subsection{Building}
As the project makes heavy use of the ECMAScript 2020\footnote{\url{https://www.ecma-international.org/publications/standards/Ecma-262.htm}} syntax, it is transpiled using the Babel cross-compiler\footnote{\url{https://babeljs.io}} to a more downwards-compatible JavaScript version when building the project.

\newpage
\subsection{Data Structures}
All join problems can be defined using the following JSON structure:
\begin{minted}{json}
[
    {
        problemID: number,
        problemNeighbors: {
            string: string
        },
        problemNumberOfRelations: number,
        problemRelations: [
            relationCardinality: number,
            relationName: string,
            relationPID: number,
            relationRID: number
        ],
        problemSelectivities: {
            string: number
        }
    }
]

\end{minted}

\subsection{Deployment}

a web application–the user does not have to update the application himself. 
The duty of updating the application is on the host's side.