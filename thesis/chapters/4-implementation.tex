\section{Implementation}
\label{sec:implementation}

\subsection{API Requirements and Design Goals}
\label{subsub:api-requirements-goals}

\paragraph{Orthogonality}
Most importantly though, the property  of orthogonality 
as defined by Eric S. Raymond in \textit{The Art of Unix
Programming} \cite{raymond2003compactness} has to be fulfilled for the visualization code with regards to the algorithm implementation's code, i.e. it should not change its behavior or state.

\paragraph{Declarativity} One of the main goals of this framework is to minimize 
the amount of code or configuration that has to be added to exisiting algorithms' implementations in order to make use of the visualization for its intended purpose in exploration or debugging. The visualization should therefore be specified as declaratively as 
possible, with the user specifying \textit{only} the relevant parameters–while minimizing the amount of required boilerplate code.

This allows for many side-effects like reusability, mainta

\paragraph{Separation of Concerns}
With the separation of the complex algorithm and user interface/visualization logic in mind is to keep the amount code required to create new visualizations as minimal as possible. 
Thus, we organize their architecture in a client-server 
relationship, whereby the client specifies the join 
problem parameters and configuration and the server can
generate a client-interpretable output to work with.

The advantages of this approach are multifold:
\begin{itemize}
    \item The server part and UI part respectively can be
          updated or even completely exchanged without 
          changing the other
    \item More complex join ordering algorithms can be 
          performed on a (more powerful) server
    \item The above allow us to run the application 
          within a web application without sacrificing 
          the performance by limiting ourselves with a 
          slow, interpreting JavaScript engine, but being 
          able to run compiled and thus more optimized and therefore faster 
          code
    \item (Allows us to formally specify interfaces between
          client and server) (Abbildung malen)
\end{itemize}

\paragraph{Flexibility}
The visualization tools should be abstract enough that it is not catered towards a specific algorithm, but can used to visualize a wide range of algorithms, especially ones it has not been explicitly programmed for.  

Since there is a large number of data structures an algorithm can use for its implementation we limit ourselves to procedures involving a query graph with information about its number of relations, the query graph type as specified in \ref{subsub:query-types}, and its cardinalities and selectivities.
This limitation is necessary to enfore a manageable scope for creating a minimum viable product.

More pragmatically, this means that the algorithm has to 
conform to the \texttt{Visualizable} Go type we define as 
follows:

\begin{figure}[H]
    \begin{minted}{go}
type Visualizable func(QG QueryGraph, JTC JoinTreeCreator) *Tree
    \end{minted}
    \caption{\texttt{Visualizable} type conformance requirement}
    \centering
\label{fig:visalizable}
\end{figure}

\begin{example}
This way, a Go function \texttt{DPccp} that conforms to this type can be instructed to produce the output  log necessary for the visualization by simply calling 

\begin{minted}{go}
visualize(DPccp, QG, JTC)
\end{minted}
instead of the algorithm itself. The \texttt{visualize} function then automatically sets the appropriate flags to enable visualization output and can allow the algorithm to run without the visualization code and thus no memory overhead.

\end{example}

\paragraph{High signal-to-noise ratio} 
In our implementation we are using code folding, enabled
through the orthogonality property as mentioned in 
section \ref{subsub:api-requirements-goals}. This allows
us to completely show or hide the visualization code 
when required. Thus, we can achieve an inobtrusive
debugging tool that can be written alongisde an 
algorithm's implementation code and doesn't decrease
its signal-to-noise ratio.


\subsection{Architecture}

The following figure \ref{fig:sequence-diagram} illustrates a sequence diagram for the client-server architecture used for this project. At first, we retrieve all avaialable algorithms, denoted by $A$, from the server. This is used to asynchronously generate the available options in the algorithm selection picker. Retrieving the algorithms from the server has the advantage that new algorithm visualizations can be added server-side without changing, re-installing or re-deploying any client code.

Next up, the user can select all the parameters for the visualization. The first parameter is the algorithm $a \in A$ to be visualized. Moreover, the query graph type $qg \in {\text{\{``chain'', ``star'', ``cycle'', ``tree'', ``moerkotte''\}}}$ can be specified. Contrary to the algorithm, the query type is not fetched from the server as the client implementation explicitly has to know the available query graphs at build time, since they are used for calculating the position of different query graph nodes as detailed in section \ref{subsub:query-graphs}. 

Furthermore, the number of relations in the query graph $n$ has to be specified, whereby $n \in [3,10]$. The upper boundary for $n$ is set to 10 to prevent calculations that are too complex and might be set even lower for algorithms other than \texttt{DPccp}, where the execution time and produced output is even larger.

Depending on the previously selected parameters the algorithm's steps are fetched upon clicking the ``Recalculate'' button. This will trigger another HTTP server request that transmits the algorithm (:a), number of relations (:r) and graph type (:g) as path parameters.

As the server receives this request, it calls the corresponding algorithm by using the \texttt{visualize()} function specified in section TODO: ref. This produces a JSON response containing the query graph $QG$ with information about cardinalities, selectivities, and neighbors, as well as the algorithm's generated visualization steps $S$.

Upon receiving this response, the client renders the query graph including labels for cardinalities and selectivities, and edges for connected query graph nodes. Moreover, the algorithm's steps are rendered in a hierarchical table.

As the rendering completes, the number of total steps |S| is calculated and the user is presented with two buttons to move to either the next or the previous step. After doing so, the user interface, including the query graph and the variable table, are rerendered accordingly.

\begin{figure}[H]
    \centering
    \begin{sequencediagram}
        \def\unitfactor{0.9}
        \newthread{A}{Client (GUI)}{}
        \newinst[8]{B}{Server}{}
        \begin{call}{A}{Render UI}{A}{}
        \end{call}
        \begin{call}{A}{GET /api/algorithms}{B}{available algorithms $A$}
        \end{call}
        \begin{call}{A}{Render algorithm picker}{A}{}
        \end{call}
        \begin{sdblock}{loop}{}
            \begin{call}{A}{Choice of algorithm, query graph type, number of relations}{A}{}
            \end{call}
            \begin{call}{A}{GET /api/algorithm/:a/relations/:r/graphType/:g}{B}{query graph $QG$, algorithm steps $S$}
            \end{call}
            \begin{call}{A}{Render query graph, algorithm steps}{A}{}
            \end{call}
            \begin{sdblock}{loop}{}
                \begin{call}{A}{User moves to next or previous step}{A}{Rerender UI}
                \end{call}
            \end{sdblock}
        \end{sdblock}
    \end{sequencediagram}
    \caption{UML sequence diagram of the client-server communication}
    \label{fig:sequence-diagram}
\end{figure}


\subsection{Choice of Technology}

For the implementation of both client and server code we
want to maximize

\begin{enumerate}[label={(\roman*)}]
    \item \label{itm:accessibility} \textbf{Accessibility}, a minimal amount of effort required to access the application
    \item \label{itm:maintainability} \textbf{Maintainability}, a high degree of technological similarities with related projects 
    \item \label{itm:testedness} \textbf{Tried-and-testedness}, a large user base and well-received reference applications
\end{enumerate}

In order to maximize \ref{itm:accessibility} we create an application that runs in a web browser and doesn't need any user-initiated installation or update procedures.
With the 2017 released WebAssembly (Wasm)\footnote{\url{https://webassembly.org}} portable byte-code standard still being in its infancy at around 90\% browser support\footnote{\url{https://caniuse.com/wasm}} and being recommended by the World Wide Web Consortium not until December 5th 2019\footnote{\url{https://www.w3.org/TR/wasm-core-1/}}, this leaves us with JavaScript as the only viable language for the client-side code. 

Additionally, in order to modularize the user interface and adding the ability to import existing user interface components, we use \texttt{React.js}\footnote{\url{https://reactjs.org}}, as of writing this thesis the most popular JavaScript library for building user interfaces.
%\footnote{Ranked by number of stars on \url{https://github.com/search?l=JavaScript&o=desc&q=language%3AJavaScript&s=stars&type=Repositories}}.
% search?l=Go&o=desc&q=http&s=stars&type=Repositories

For the server code and algorithm implementations we use the Go programming language and \texttt{gin-gonic}\footnote{\url{https://github.com/gin-gonic/gin}}, which at the time of writing this thesis is the most-starred HTTP framework on GitHub\footnote{\url{https://github.com/}}.

%Thus, we resort to modern, but well-adopted languages and libraries.
%The software should be built with technologies that prepare well for continued development and maintenance.

\subsection{Tooling}

\subsubsection{Join Problem Generator}
We represent the join problems in a JSON format as defined in section \ref{subsub:join-problem-data-structure}. 
Some JSON files are taken from the supplementary material to the ``IE 630 Query Optimization'' course exercises at the University of Mannheim\footnote{\url{https://www.wim.uni-mannheim.de/moerkotte/teaching/courses/query-optimization/}}
, which contains examples for join problems with a query type $QT \in $ \{Chain, Clique, Cycle, Star\} and a number of relations $n \in [2,10]$.

Additionally, we create a join problem generator for complete binary trees with a degree $k$ and a number of relations $n$.

Since selectivities are symmetrical, they are stored once and once only in the JSON. This means, if an edge ($R_0, R_1$) in the query graph exists, we don't store ($R_1, R_0$) explicitly as well. However, we store the selectivities in a breadth-first manner starting from $R_0$, i.e. the edge always points from the relation with the smaller index to the larger one. More formally, we can say that the following implication holds:

\begin{equation}
    \forall (R_i, R_j) \in E \implies i < j
\label{eq:bfs-implication}
\end{equation}

\setlength{\belowdisplayskip}{10pt}
\setlength{\belowdisplayshortskip}{50pt}
\setlength{\abovedisplayskip}{10pt}
\setlength{\abovedisplayshortskip}{10pt}

\paragraph{Complete Binary Tree}
\subparagraph{Selectivities}
Since implication \ref{eq:bfs-implication} holds, we only need to calculate selectivities for nodes which have children.
Thus, first of all, we calculate the number of vertices with children $n_c$, i.e. the number of relation-pairs for which selectivity entries exist in the resulting JSON file.
In the case of complete $k$-ary trees, when enumerating the relations in a breadth-first manner, this number is given by 
\begin{equation}
    n_{c} = \#\{v \in V \vert \text{hasChildren}(v)\} = \lceil\frac{n}{k}\rceil
\end{equation}
Accordingly, we specify the set of vertices with children $V_c$, starting with index 0.
\begin{equation}
   V_c := \{v_i\}_{i=0}^{n_{c}-1}
\end{equation} 

\begin{note}
Let $I$ be the index set for all relations
\begin{equation}
    I = \{0,\ldots,n-1\}
\end{equation}
Then, since our $k$-ary tree is complete and breadth-first, we can infer that if a node at index $i$ does not have children, any node at a larger index $j$ doesn't either. Likewise, any node with an index $i < n_c$ does have children. 

\begin{equation}
    \forall i,j \in I, i < j \colon v_i \notin V_c \implies v_j \notin V_c
\end{equation}
\begin{equation}
    \forall i < n_c \colon \exists v_i \in V_c
\end{equation}

This is why it suffices to consider nodes with index $i < n_c$ when generating selectivity entries.
\end{note}

Furthermore, we need to determine the level $l(v_i)$ of each node $v_i$ in the tree. For its respective index $i$, this is given by
\begin{equation}
    l_i = l(v_i) = log_k(i+1)
\end{equation}

Next up, we then define the set of vertices in the levels below $l_i$, denoted $V_{l_{i<j}}$, as follows:
\begin{equation}
    V_{l_{j<i}} := \{v_j \in V \vert l(v_j) < l(v_i) \}
\end{equation}

Now, we need the cardinality of this set, defined as $n_{j<i}$ to get the number of nodes on all level smaller than $l_i$. This can be efficiently calculated using the following equation:
\begin{equation}
     n_{j<i} := \vert V_{l_{j<i}}\vert = 2^{l(v_i) + 1} - 1
\end{equation}

This allows us to calculate the number of children for each level $l(v_i)$.
\begin{equation}
    \hat{l}_i = \min(2^{l(v_i) + 1}, n - n_{j<i})
\end{equation}

\begin{note}
Note that $l(v_i)$ is not necessarily equal to the width of the level in a full $k$-ary tree, which is given by $2^{l(v_i) + 1}$, since some leaf nodes could be missing in the bottom level of the tree.
\end{note}

Furthermore, we calculate the ``column'' $c_i$ of $v_i$, given by
\begin{equation}
    c_i = i - 2^{l(v_i)} + 1
\end{equation}

This lets us calculate the number of children $\hat{n}_i$ for a node at index $i$:
\begin{equation}
    \hat{n}_i = \min(k, \hat{l} - c_i * k)
\end{equation}

\begin{note}
    This $\hat{n}_i$ is equal to $k$ for most internal nodes in a sufficiently large tree. However, if $log_k(n) \notin \mathbb{N}$ there could be an internal node with $\hat{n}_i < k$, i.e. not all possible leaf nodes for the last node with childrens are filled.
\end{note}

Now we can define the neighbors of $v_i$ as
\begin{equation}
    \mathbb{N}_i = \{ x_{i, j} \}_{j=0}^{\hat{n}_i - 1} 
\end{equation}

We also calculate the lowest index of the node's neighbors $j_{min\mathbb{N}_i}$. (TODO: Not entirely sure about this notation).
\begin{equation}
    j_{min\mathbb{N}_i} = ik + 1
\end{equation}

For each neighbor $x_{i, j} \in \mathbb{N}_i$ we then calculate the offset for its $j$ index.
\begin{equation}
    \tau_{i,j} = j\mod{k}
\end{equation}

The current index of the neighbor is thus given by
\begin{equation}
    \sigma_{i,j} = j_{min\mathbb{N}_i} + \tau_{i,j}
\end{equation}

Finally, we can set all $x_{i, j} \in \mathbb{N}_i$ as $\sigma{i,j}$
\begin{equation}
    x_{i, j} = \sigma_{i,j}
\end{equation}

Let us now define the set of all neighbors $\mathbb{N}$ for all indices $i < n$:
\begin{equation}
    \mathbb{N} = \{\mathbb{N}_i\}_{i=0}^{n-1}
\end{equation}

We can now set a selectivity for each $(x_i,j) \in \mathbb{N}_i \in \mathbb{N}$, where the selectivity $s_{i,j}$ is a random varialbe following a uniform distribution over the set $\mathbb{R} \cap [0,1]$.

A Go implementation example of the symmetrical neighbor calculation is illustrated in algorithm \ref{alg:symNeighbors}. The assignment of selectivities to these entries is trivial and thus not illustrated here, but can be found in the supplementary code.

\begin{note}
The variable names in algorithm \ref{alg:symNeighbors} correspond to the names used in the previously used mathematical definitions. In the supplementary code to this thesis longer, more descriptive variable names are used.
\end{note}

%TODO: Replace \texttt{log2_{64}} with \texttt{math.Log}

\begin{algorithm}
\begin{minted}{go}
symNeighborEntry := func(i uint, n_c uint) string {
    l_i := uint(log2_64(uint64(i + 1)))
    n_j_leq_i := uint(1<<(l_i+1) - 1)
    l_hat_i := min(1<<(l_i+1), n-n_j_leq_i)
    c_i := i - 1<<(l_i) + 1
    n_hat_i := min(k, l_hat_i-c_i*k)

    N := make([]string, n_c)
    j_minN_i := i*k + 1
    for j := range N {
        tau_ij := uint(j) % k
        sigma_ij := j_minN_i + tau_ij
        s := strconv.FormatUint(uint64(sigma_ij), 10)
        N[j] = s
    }
    return strings.Join(N, ",")
}
symNeighbors := func() map[uint]string {
    dict := map[uint]string{}
    n_c := math.Floor(float64(n) / float64(k))
    for i := uint(0); float64(i) < n_c; i++ {
        dict[i] = symNeighborEntry(i, n)
    }
    return dict
}
problemNeighbors := symNeighbors()
\end{minted}
\caption{Go implementation to calculate symmetrical neighbors. \texttt{k} is the degree of the tree and \texttt{n} the number of relations. Both must be set before using them in the above lambda expressions (Go function literals).}
\label{alg:symNeighbors}
\end{algorithm}

\subparagraph{Neighbors}
The set of neighbors does not make use of the symmetry of selectivities and thus stores all neighbors for each relation explicitly. Thus, we append the parent node $\rho_i$ to the set of neighbors $\mathbb{N}$ calculated for the selectivity generation. This assignment is trivial and thus the algorithm is not provided here.

% We then join all $$
% Joining x_{i, j} into array for each i
\subparagraph{Relations}
Specifying the relations $R$ is trivial. 

\begin{equation}
R := \{r_i\}_{i=0}^{n-1}
\end{equation}
where the cardinality $c_{r_i}$ of each relation $r_i$ is a random variable following uniform distribution over the set $\mathbb{R} \cap [0,10000]$.

Likewise, the creation of the \texttt{relations} slice is trivial and therefore we don't illustrate the algorithm here.

\subsection{Data Structures}

\subsubsection{Join Ordering}
In order to parse and represent a join problem from a JSON file we use the already existing data structures \texttt{JSONRelation} and \texttt{JSONJoinProblem} from \texttt{joinproblemparser.go}. Both are implicitly marshalled and unmarshalled when converted from or to JSON because of the built-in JSON encoding and decoding for custom data structures using the Go standard library's \texttt{encoding/json} package\footnote{\url{https://golang.org/pkg/encoding/json/}}. However we add \texttt{struct} tags to explicitly set the corresponding JSON keys.

TODO: Remove "Relation" prefix here and in code
\begin{code}
\begin{minted}{go}
type JSONRelation struct {
    RelationCardinality float64 `json:"relationCardinality"`
    RelationName        string  `json:"relationName"`
    RelationPID         uint    `json:"relationPID"`
    RelationRID         uint    `json:"relationRID"`
}
\end{minted}
\caption{\texttt{JSONRelation} type}
\end{code}
\vspace{0.8cm}

TODO: Remove "Problem" prefix here and in code
\begin{code}
\begin{minted}{go}
type JSONJoinProblem struct {
    ProblemID                uint               `json:"problemID"`
    ProblemNeighbors         map[uint]string    `json:"problemNeighbors"`
    ProblemNumberOfRelations uint               `json:"problemNumberOfRelations"`
    ProblemRelations         []JSONRelation     `json:"problemRelations"`
    ProblemSelectivities     map[string]float64 `json:"problemSelectivities"`
}
\end{minted}
\caption{\texttt{JSONJoinProblem} type}
\end{code}
\vspace{0.8cm}

We'll take the \texttt{QueryGraph} \texttt{struct} from \texttt{infra.go} and add a \texttt{map} to a \texttt{uint} slice of neighbors for each relation, which itself is represented as a \texttt{uint} bitvector and used as the map's key.

\begin{code}
\begin{minted}{go}
type QueryGraph struct {
	R []uint           `json:"relationCardinalities"`
	S map[uint]float64 `json:"selectivities"`
	N map[uint][]uint  `json:"neighbors"` // Added neighbor map
}
\end{minted}
\caption{\texttt{QueryGraph} type}
\end{code}
\vspace{0.8cm}



Also, we introduce a struct for representing csg-cmp-pairs.
\begin{code}
\begin{minted}{go}
type CsgCmpPair struct {
    S1 uint
    S2 uint
}
\end{minted}
\caption{\texttt{CsgCmpPair} type}
\end{code}
\vspace{0.8cm}

\subsubsection{Visualization}

For the visualization we define some data structures used to declaratively specify algorithms, their visualization routines and their atomic components, such as visualization steps and the subroutine's configured observed relations. 

\paragraph{Algorithm}

First, we specify an \texttt{Algorithm} type that can be returned to the client when requesting the list of algorithms which can be visualized. Each algorithm contains a \texttt{Label} string describing the algorithm, e.g. ``DPccp'' as well as a \texttt{Value} string giving the algorithm's identifier for server requests. This \texttt{Value} string is by convention a camel-cased name of the algorithm starting with a lower-case character, whereas the \texttt{Label} can also contain special or whitespace characters. This differentiation is made to allow human-readable algorithm names as well as URL-encodable identifiers for the API's path parameter.

\begin{code}
\begin{minted}{go}   
type Algorithm struct {
    Label string `json:"label"`
    Value string `json:"value"`
}
\end{minted}
\caption{\texttt{Algorithm} type}
\end{code}
\vspace{0.8cm}

\paragraph{Variable Table}

We want to keep track of a set of user-defined relation variables to aid understanding the algorithm's state's history. For this purpose, we first introduce the \texttt{ObservedRelation} type that can be used to specify which sets of relations should be observed. It contains an \texttt{Identifier} string that corresponds to the variable's name, as well as a \texttt{Color} field, which allows to visually distinguish different variables. This is specified using a \texttt{color.RGBA} struct from the \texttt{image/color} package, which can be found in Go's standard library.

\begin{code}
\begin{minted}{go}
type ObservedRelation struct {
    Identifier string     `json:"identifier"`
    Color      color.RGBA `json:"color"`
}
\end{minted}
\caption{\texttt{ObservedRelation} type}
\end{code}
\vspace{0.8cm}

Furthermore, we introduce a \texttt{VariableTableEntry} type that can be specified individually for each observed relation, in each atomic visualization step. In our application, this entry is a \texttt{uint} slice depicting the current variable's set of relations in bitvector representation.

\begin{code}
\begin{minted}{go}
type VariableTableEntry []uint
\end{minted}
\caption{\texttt{VariableTableEntry} type}
\end{code}
\vspace{0.8cm}

TODO: Rename in code
These variable table entries are stored in a \texttt{VariableTableRow} map. A string identifying the relation that corresponds to the \texttt{Identifier} in the \texttt{ObservedRelation} as the key of the map.

\begin{code}
\begin{minted}{go}
type VariableTableRow map[string]VariableTableEntry
\end{minted}
\caption{\texttt{VariableTableRow} type}
\end{code}
\vspace{0.8cm}

\paragraph{GraphState}

The state of the query graph drawn in the user interface must be represented in order to unambigiously specify its configuration. We want to be able to give each node in the graph a color that corresponds to the \texttt{Color} field of the \texttt{ObservedRelation}. Hence, we first define a \texttt{NodeColor} type storing the \texttt{Index} of the relation, as well as the respective \texttt{Color}.

\begin{code}
\begin{minted}{go}
type NodeColor struct {
    NodeIndex uint       `json:"nodeIndex"`
    Color     color.RGBA `json:"color"`
}
\end{minted}
\caption{\texttt{NodeColor} type}
\end{code}
\vspace{0.8cm}

The graph state is thus a slice containing all node colors. Since we explicitly assign a color to each node in every visualization step, the length of the slice is therefore equal to the number of relations $n$ in the graph.

\begin{code}
\begin{minted}{go}
type GraphState struct {
    NodeColors []NodeColor `json:"nodeColors"`
}
\end{minted}
\caption{\texttt{GraphState} type}
\end{code}
\vspace{0.8cm}

\paragraph{Steps and Routines}

Finally, the data structures defined beforehand can be used to construct higher-level constructs such as the \textit{atomic} visualization step, represented as a \texttt{VisualizationStep} type. This type contains the \texttt{GraphState} for the current step, as well as the variables stored as a \texttt{VariableTableRow}. In order to identify a step in equivalence checks, we also assign a \texttt{UUID}, which is generated using Google's own \texttt{github.com/google/uuid}\footnote{\url{https://github.com/google/uuid}} package. This UUID is a 128 bit UUID based on RFC4122\footnote{\url{https://tools.ietf.org/html/rfc4122}}, which can be easily converted to a \texttt{string} using the \texttt{uuid.New().String()} function. This \texttt{UUID} is necessary as the reference to the exact instance of a struct is lost when marshalling it into JSON.

\begin{code}
\begin{minted}{go}
type VisualizationStep struct {
    GraphState GraphState       `json:"graphState"`
    Variables  VariableTableRow `json:"variables"`
    UUID       string           `json:"uuid"`
}
\end{minted}
\caption{\texttt{VisualizationStep} type}
\end{code}
\vspace{0.8cm}

Finally, we have defined all the types necessary to define a high-level \texttt{VisualizationRoutine}. This routine contains a \texttt{Name} field to describe its corresponding algorithm, e.g. ``EnumerateCsg'', as well as fields to store an \texttt{ObservedRelation} slice to specify the sets of relations that should be kept track of for this routine. Furthermore, we store a slice of all the routines steps. This \texttt{Steps} slice is genericly implemented using a pointer to Go's \texttt{interface{}} type, as it can be filled with either an atomic \texttt{VisualizationStep} or another \texttt{VisualizationRoutine}. This allows for subroutines or even recursive calls of other subroutines and creates a hierarchical structure of all steps in the visualization's JSON representation. 

\begin{code}
\begin{minted}{go}
type VisualizationRoutine struct {
    Name              string             `json:"name"`
    ObservedRelations []ObservedRelation `json:"observedRelations"`
    Steps             []*interface{}     `json:"steps"`
}
\end{minted}
\caption{\texttt{VisualizationRoutine} type}
\end{code}
\label{datastructure:visualization-routine}
\vspace{0.8cm}

This \texttt{VisualizationRoutine} is transferred as a top-level object when making a server request and thus we don't need to specify any more data structures.

\subsection{Server Implementation}

\subsubsection{Utility Functions}


In order to provide a starting point for the algorithm implementation, we use a modified and extended version of the \url{infra.go} and \url{joinproblemparser.go} file provided for the exercises in the Query Optimization (CS50x) course \texttt{} at the Chair of Practical Computer Science III at the University of Mannheim.

Modifications include:
\paragraph{Simplification of \texttt{SetMinus}}
We simplify the implementation for the \texttt{SetMinus} operation by omitting the use of a temporary helper variable for the result calculation. This is equivalent as it produces exactly the same assembly output as the old, commented-out code\footnote{using amd64 gc 1.15 compiler, compiled on \url{https://godbolt.org}}.

\begin{minted}{go}
func SetMinus(S1 uint, S2 uint, length uint) uint {
    mask := uint((1 << length) - 1)
    // temp := S1 & S2
    // temp = ^temp
    // temp &= mask
    // return S1 & temp
    return S1 & ^S2 & mask
}
\end{minted}

\paragraph{Addition of \texttt{PowerSet}}
For the implementation of \texttt{DPccp} we need an additional function that returns not just the subsets of an (\texttt{unsigned int}) bit vector $S$ as does \texttt{Subsets}, but its power set excluding the empty set, $\mathcal P(S)\setminus\{\varnothing\}$.

\begin{minted}{go}
func PowerSet(S uint) []uint {
    subsets := Subsets(S)
    if len(subsets) == 1 && subsets[0] == S {
        return subsets
    }
    return append(subsets, S)
}
\end{minted}

We add some additional helper functions:
\begin{minted}{go}   
func contains(s []uint, e uint) bool {
    for _, a := range s {
        if a == e {
            return true
        }
    }
    return false
}    
\end{minted}

First, we'll add a method to retrieve the neighborhood for a subset of relations $S$.

\begin{algorithm}[H]
\begin{minted}{go}
func Neighbors(QG QueryGraph, S uint) uint {
    indexes := IdxsOfSetBits(S)
    result := uint(0)
    for _, index := range indexes {
        for _, neighbor := range QG.N[index] {
            result = result | (1 << neighbor)
        }
    }
    n := uint(len(QG.R))
    return SetMinus(result, S, n)
}
\end{minted}
\caption{\texttt{Neighbors} function for a set of relations}
\end{algorithm}

\subsubsection{Visualization}

For the visualization we create a \texttt{visualization} package to handle all operations related to the visualization itself. Here, we keep track of an exported \texttt{VisualizationOn} Boolean variable to indicate whether the visualization code should be executed or not. Furthermore, we define two local variables:
\begin{minted}{go}
var routines = []*VisualizationRoutine{}
var stack = []*VisualizationRoutine{}
\end{minted}
Both are slices of pointers to the \texttt{VisualizationRoutine} type defined in figure \ref{datastructure:visualization-routine}. Both fulfill different purposes: the former keeps track of consecutively executed visualization routines, whereas the latter stores the current execution stack of the current routine, which allow for stacked or recursive routine calls. In order to enforce modularity, those variables cannot be changed directly, but have to use exported functions to start new routines or add atomic visualization steps.

For this purpose, we define the following functions:
\begin{minted}{go}
func visualize(visualization Visualizable, QG QueryGraph, JTC JoinTreeCreator) []*VisualizationRoutine
\end{minted}

This \texttt{visualize} function can be used to call a function that conforms to the \texttt{Visualizable} type as defined in \ref{fig:visalizable}, one example being our implementation of \texttt{DPccp}. This will automatically set the \texttt{VisualizationOn} Boolean to \texttt{true}, execute the algorithm, and set it back to its previous value after finishing the execution. Moreover, it will return all the emitted routines as a \texttt{*ViszalizationRoutine} slice and reset all local variables after execution.

\begin{example}
    An implementation of the \texttt{DPccp} algorithm defined through the following function
    \begin{minted}{go}
func DPccp(QG QueryGraph, JTC JoinTreeCreator) *Tree
    \end{minted}
    can be visualized by calling
    \begin{minted}{go}
visualize(DPccp, QG, JTC)
    \end{minted}
    instead of directly calling
    \begin{minted}{go}
DPccp(QG, JTC)
    \end{minted}

Notice that the function definition of \texttt{DPccp} doesn't have to be changed from its original implementation at all here. Thus, potenitially already existing calls to \texttt{DPccp} don't have to be modified.
\end{example}

Within the function implementation we can now use the exported visalization functions to start new routines and add new visualization steps. Before adding the first step, we have to call the following function at least once: 
\begin{minted}{go}
func startVisualizeRoutine(routine *VisualizationRoutine)
\end{minted}

As the name implies, this function can be used to start a new visualization routine. This will add a new routine to the \texttt{routines} slice if the stack empty or push it to the stack if there's already a function being executed. Notice that whether the function is introduced as a consecutive routine or added to the call stack is determined by the \texttt{visualization} package implicitly, for the convenience of the package's user. An example how this call stack could look like for \texttt{DPccp} at a certain point in its execution is shown in figure \ref{fig:execution-stack}.

\begin{figure}[H]
\begin{tikzpicture}[stack/.style={rectangle split, rectangle split parts=#1,draw, anchor=center}]
    \node[stack=5]  {
        \nodepart{one}EnumerateCsgRec
        \nodepart{two}EnumerateCsgRec
        \nodepart{three}EnumerateCsgRec
        \nodepart{four}EnumerateCsg
        \nodepart{five}DPccp
    };
    \end{tikzpicture}
\caption{Sample excution stack for \texttt{DPccp}}
\label{fig:execution-stack}
\end{figure}

After starting a visualization routine we either start a new routine that is added onto the stack or add a visualization step using the following function, which takes a query graph and a \texttt{VariableTable} as its parameters.

\begin{minted}{go}
func addVisualizationStep(QG QueryGraph, relations VariableTable)
\end{minted}

The implementation of this function will automatically construct a new \texttt{GraphState} from information in the passed \texttt{VariableTable} by retrieving its color configuration. This makes sure the observed relations and their corresponding nodes in the query graph have the same color without the need to specify it separately. Furthermore, a \texttt{UUID} is assigned to the visualization step, which used for efficient equivalence checks in the client implementation.

\begin{algorithm}[H]
\begin{minted}{go}
func addVisualizationStep(QG QueryGraph, relations VariableTable) {
    n := uint(len(QG.R))
    nodeColors := []NodeColor{}
    currentStackIndex := len(stack) - 1

    // Create graph state
    observedRelations := stack[currentStackIndex].ObservedRelations
    for i := n - 1; int(i-1) >= -1; i-- {
        for _, relation := range observedRelations {
            relationIndexes := relations[relation.Identifier]
            if contains(relationIndexes, i) {
                color := relation.Color
                nodeColor := NodeColor{NodeIndex: i, Color: color}
                nodeColors = append(nodeColors, nodeColor)
            }
        }
    }
    graphState := GraphState{NodeColors: nodeColors}
    uuid := uuid.New().String()
    step := &VisualizationStep{GraphState: graphState, 
                               Variables: relations, 
                               UUID: uuid}
    currentRoutine := stack[currentStackIndex]
    var v interface{}
    v = step
    currentRoutine.Steps = append(currentRoutine.Steps, &v)
}
\end{minted}
\caption{Implementation of \texttt{addVisualizationStep} function to create a new atomic visualization step.}
\end{algorithm}

\begin{note}
For the time being, we make the assumption that only one node color can be assigned to each node in the visualized query graph. Thus, we must make sure that the sets of all observed relations are disjoint. In order to formalize this requirement we define the set of observed relations $O$. Assume there are $v$ observed variables. For a concise mathematical definition we assume variable keys are integers $i \in \mathbb{N}$ from 0 to $v-1$:
\begin{equation}
O := \{O_i\}_{i=0}^{v-1}
\end{equation}
\end{note}

Thus, the following equation must hold
\begin{equation}
\bigcap_{i=0}^{v-1}{O_i} \overset{!}{=} \emptyset
\label{eq:disjointness-observed-relations}
\end{equation}

However, the union of these sets $\bigcup_{i=0}^{v-1}{O_i}$ does not necessarily have to be equal to the set of all relations $R$, but it suffices if $O \subseteq {R}$. If $O \not= R$ case we can just fall back to a  default color. This could be handled either in the implementation of \texttt{addVisualizationStep} or, alternatively, in the client-side implementation when drawing the query graph, which is how the supplementary code handles $O \subset R$.

\subparagraph{Excursus: Impact on \texttt{DPccp} implementation}
In \texttt{EnumerateCsgRec} inside \texttt{DPccp} we want to visualize the sets of relations $S$ and $X$, among others.  Assumption \ref{eq:disjointness-observed-relations} can then be fulfilled if we slightly change the definition of $\mathcal{B}_i$, so that inside \texttt{EnumerateCsgRec} $S \cap X = \emptyset$ holds. 

\begin{theorem}
$\mathcal{B}_i := \{v_j\vert j \leq i\}$ can be redefined to $\mathcal{B}_i := \{v_j\vert j < i\}$ without changing the emits of the algorithm, i.e. in every call of \texttt{EnumerateCsgRec} we can still assume that $v_i \notin S'$.
\end{theorem}

\begin{proof}
Let us start at the \texttt{EnumerateCsgRec} call inside \texttt{EnumerateCsg}. The variables $S := \{v_i\}$ and $X := \mathcal{B}_i$ are used as parameters to \texttt{EnumerateCsgRec} and therefore, inside this call, $v_i \in S$. This will also hold for all recursive calls as the recursive set, denoted $S_{rec}$, will be defined as  $S_{rec} := S' \cup S \supseteq S$. From the definition of $\mathbb{N}(S)$ it follows that $\forall v_j \in S \colon v_j \notin \mathbb{N}(S)$ and thus $v_i \notin \mathbb{N}(S)$. Since $N := \mathbb{N}(S)\setminus X$ it follows that $N \subseteq \mathbb{N}(S)$, from which it transitively follows that $v_i \notin N$. Thus, since $S' \subseteq N$, it follows that $v_i \notin S'$.

\end{proof}

\paragraph{Usage}

The usage of this algorithm is also demonstrated in the next section, \ref{subsub:implementation-dpccp}, where it is used in combination with the \texttt{EnumerateCsg} algorithm. We visualize the entire algorithm in the supplemented code.

\subsubsection{DPccp}
\label{subsub:implementation-dpccp}



\begin{algorithm}
\begin{minted}{go}
func EnumerateCsg(QG QueryGraph) []uint {
    n := uint(len(QG.R))
    subgraphs := []uint{}
    for i := n - 1; i < n; i-- {
        v := uint(1 << i)
        subgraphs = append(subgraphs, v)
        B := uint(1<<(i+1) - 1)
        recursiveSubgraphs := EnumerateCsgRec(QG, v, B)
        subgraphs = append(subgraphs, recursiveSubgraphs...)
    }
    return subgraphs
}
\end{minted}
\caption{Go implementation of \texttt{EnumerateCsg}}
\label{alg:enumeratecsg}
\end{algorithm}

\begin{algorithm}
\begin{minted}{go}
func EnumerateCsgRec(QG QueryGraph, S uint, X uint) []uint {
    n := uint(len(QG.R))
    Neighbors := Neighbors(QG, S)
    N := SetMinus(Neighbors, X, n)

    subgraphs := []uint{}

    for _, SPrime := range PowerSet(N) {
        if SPrime == 0 {
            continue
        }
        SuSPrime := S | SPrime
        subgraphs = append(subgraphs, SuSPrime)
    }
    for _, SPrime := range PowerSet(N) {
        if SPrime == 0 {
            continue
        }
        SuSPrime := S | SPrime
        XuN := X | N
        recursiveSubgraphs := EnumerateCsgRec(QG, SuSPrime, XuN)
        subgraphs = append(subgraphs, recursiveSubgraphs...)
    }
    return subgraphs
}
\end{minted}
\caption{Go implementation of \texttt{EnumerateCsgRec}}
\label{alg:enumeratecsgrec}
\end{algorithm}

\begin{algorithm}
\begin{minted}{go}
func EnumerateCmp(QG QueryGraph, S1 uint) []CsgCmpPair {
    minS1 := MinUintSetBitIndex(S1)
    BminS1 := uint(1<<minS1) - 1

    X := BminS1 | S1
    n := uint(len(QG.R))
    neighbors := Neighbors(QG, S1)
    N := SetMinus(neighbors, X, n)

    subgraphs := []CsgCmpPair{}
    setBits := IdxsOfSetBits(N)
    for i := len(setBits) - 1; i >= 0; i-- { // Descending
        v := setBits[i]
        pair := CsgCmpPair{Subgraph1: S1, Subgraph2: 1 << v}
        subgraphs = append(subgraphs, pair)

        Bi := uint(1<<v - 1)
        recursiveComplements := EnumerateCsgRec(QG, 1<<v, X|(Bi&N))
        for _, S2 := range recursiveComplements {
            pair := CsgCmpPair{Subgraph1: S1, Subgraph2: S2}
            subgraphs = append(subgraphs, pair)
        }
    }
    return subgraphs
}
\end{minted}
\caption{Go implementation of \texttt{EnumerateCmp}}
\label{alg:enumeratecmp}
\end{algorithm}

\begin{algorithm}
\begin{minted}{go}
func DPccp(QG QueryGraph, JTC JoinTreeCreator) *Tree {
    n := uint(len(QG.R))
    bestTree := make([]*Tree, 1<<n)
    for i := uint(0); i < n; i++ {
        card := float64(QG.R[i])
        tree := &Tree{card, 1 << i, nil, nil, 0, nil}
        bestTree[1<<i] = tree
    }
    subgraphs := EnumerateCsg(QG)
    csgCmpPairs := []CsgCmpPair{}
    for _, subgraph := range subgraphs {
        subgraphCsgCmpPairs := EnumerateCmp(QG, subgraph)
        csgCmpPairs = append(csgCmpPairs, subgraphCsgCmpPairs...)
    }
    for _, csgCmpPair := range csgCmpPairs {
        S1 := csgCmpPair.Subgraph1
        S2 := csgCmpPair.Subgraph2
        S := S1 | S2

        p1 := bestTree[S1]
        p2 := bestTree[S2]

        currentTree := JTC.CreateJoinTree(p1, p2, QG)
        if bestTree[S] == nil {
            bestTree[S] = currentTree
        } else if bestTree[S].Cost > currentTree.Cost {
            bestTree[S] = currentTree
        }
        currentTree = JTC.CreateJoinTree(p2, p1, QG)
        if bestTree[S] == nil {
            bestTree[S] = currentTree
        } else if bestTree[S].Cost > currentTree.Cost {
            bestTree[S] = currentTree
        }
    }
    return bestTree[(1<<n)-1]
}
\end{minted}
\caption{Go implementation of \texttt{DPccp}}
\label{alg:dpccp}
\end{algorithm}





\subsubsection{Building}

The server code is written in Go 1.15.1\footnote{\url{https://github.com/golang/go/releases/tag/go1.15.1}} released on September 1st 2020, and using the \texttt{go1.15.1 darwin/amd64} compiler. We do not set any custom build flags.

\paragraph{Linting} The project uses the Go language linter\footnote{\url{https://github.com/golang/lint}} to ensure a common coding and documentation style with what is used at Google, the creators of Go.
All of the supplied files (\texttt{infra.go} and \texttt{joinproblemparser.go} have been modified to satisfy all linter requirements by doing some miscellaneous changes to remove warnings generated from static analysis by using the official Go linter\footnote{\url{https://github.com/golang/lint}}, e.g. by converting variable and function names from \texttt{snake\_case} to \texttt{camelCase}, or adding descriptive comments to exported functions. As these changes are mostly trivial we omit further details here.

\subsection{JSON Representation}
\label{sub:data-structures}

\subsubsection{Join Problem}
\label{subsub:join-problem-data-structure}

All join problems can be defined using the following JSON structure. ``JSON'' refers to the JavaScript Object Notation Data Interchange Format as specified in RFC 7159\footnote{\url{https://tools.ietf.org/html/rfc7159}}.
\begin{figure}
\begin{minted}{json}
[
    {
        problemID: number,
        problemNeighbors: {
            number{relationRID}: [neighbors]
        },
        problemNumberOfRelations: number,
        problemRelations: [
            relationCardinality: number,
            relationName: string,
            relationPID: number,
            relationRID: number
        ],
        problemSelectivities: {
            [s1,s2]: number
        }
    }
]
\end{minted}
\caption{JSON representation of a join problem}
\end{figure}

The names of the keys correspond to the explicitly defined keys for the \texttt{JSONJoinProblem} type in the \texttt{joinProblemParser.go} file.

\subsection{Client Implementation}

\paragraph{Pairing Function}
We use the \textit{elegant pairing function} developed by Matthew Szudzik \cite{szudzik2006elegant} for two $k_1,k_2 \in \mathbb{Z}$ in order to create a unique key for repeated elements in a nested loop with depth 2.

\begin{equation}
    \begin{aligned}
        p: \hspace{0.3cm} & \mathbb{Z} \times \mathbb{Z} \rightarrow \mathbb{N}\\
        & (k_1,k_2) \mapsto
        \begin{cases}
        k_1 * k_1 + k_1 + k_2, & \text{if}\ k_1 \geq k_2 \\
        k_2 * k_2 + k_1, & \text{otherwise}
        \end{cases}
    \end{aligned}
\end{equation}

\subsubsection{Building}
As the project makes heavy use of the ECMAScript 2020\footnote{\url{https://www.ecma-international.org/publications/standards/Ecma-262.htm}} syntax, it is transpiled using the Babel cross-compiler\footnote{\url{https://babeljs.io}} to a more downwards-compatible JavaScript version when building the project.

\paragraph{Linting}
For the client-side code we use ESLint\footnote{\url{https://eslint.org}} configured with the \texttt{"react-app"} template\footnote{\url{https://github.com/facebook/create-react-app/blob/master/.eslintrc.json}}, which itself is an extension of the \texttt{"eslint:recommended"} template as specified on \url{https://eslint.org/docs/rules/}.

\paragraph{Dependencies}

Since we're restructurig the files from the QueryCompiler exercise to increase modularity and reusability . We define all packages in a way that no cyclical dependencies exist.

\subsection{Installation and Development}

\subsubsection{Prerequisites}
In order to run the project locally some prerequisites have to be fulfilled:
\begin{itemize}
    \item The Go programming language has to be installed according to the official installation guidelines\footnote{\url{https://golang.org/doc/install}}
    \item The Go development has to be set up correctly, namely specifying the \texttt{\$GOOROOT} and \texttt{\$GOPATH} environment variables
    \item The Node Package Manager (npm) has to be installed along with the Node.js runtime\footnote{\url{https://www.npmjs.com/get-npm}}
\end{itemize}

\subsubsection{Dependencies}
\label{subsub:dependencies}

\paragraph{Client}
Client dependencies are maintained in the \texttt{package.json} file of the project's \texttt{client} directory.

\paragraph{Server}
The server-side code doesn't require any dependencies other than the Go standard libary packages as listed on \url{https://golang.org/pkg/#stdlib}.

\subsubsection{Local Development}
Once dependencies are installed as specified in section \ref{subsub:dependencies}, the project can be run locally by navigating to the project directory in a Unix shell (macOS and Linux) or PowerShell (Windows) and running

\begin{minted}{shell-session}
    $ cd client && npm install && npm start
\end{minted}
and, in another shell instance,
\begin{minted}{shell-session}
    $ cd server && go install && go run main.go
\end{minted}

This will start two locally running web servers that can be accessed via \url{https://localhost:3000/} (Client) and \url{https://localhost:8080/} (Server).


\subsection{Deployment}

One of the major benefits of using a client-server architecture is that we can host the server part on a web server or platform-as-a-service offerings such as Google's Cloud App Engine\footnote{\url{https://cloud.google.com/appengine}}.
This way, the user of the visualization neither has to install nor update the application himself, as this duty is completely on the host's side.

In the following we demonstrate how to set up the entire application stack, both client and server, in order to get from the cloned repository to a deployed application using the Google Cloud SDK\footnote{\url{https://cloud.google.com/sdk}} for the server-side application and GitHub Pages\footnote{\url{https://pages.github.com}} for hosting the frontend.

TODO: Move to appendix?
\paragraph{Prerequisites}   
    
The Google Cloud SDK must be installed using 
\begin{minted}{shell-session}
    $ curl https://sdk.cloud.google.com | bash
\end{minted}
Follow the instructions in the terminal session to continue. 

